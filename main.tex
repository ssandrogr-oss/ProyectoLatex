\documentclass[12pt]{article}
\usepackage{graphicx} % Required for inserting images

\usepackage{cite}

\usepackage{amsmath}
\usepackage{circuitikz}
\usepackage{booktabs}
\usepackage{siunitx}
\usepackage{afterpage}
\usepackage{amsmath,amssymb,amsfonts}
\usepackage{algorithmic}
\usepackage{graphicx}
\usepackage{textcomp}
\usepackage{xcolor}
\usepackage{amssymb}
\usepackage{siunitx}
\usepackage{float}
\PassOptionsToPackage{hyphens}{url}\usepackage{hyperref}
\usepackage{cleveref}
\usepackage[utf8]{inputenc}
\usepackage[right]{lineno}
\usepackage{csquotes}
\usepackage{booktabs}
\usepackage{longtable}
\usepackage{adjustbox}
\usepackage{titlesec}
\usepackage{tikz,xcolor,pgfplots}
\usepackage{amsthm}
\theoremstyle{definition}
\newtheorem{example}{Ejemplo}


\usepackage{mdframed}
\usepackage[utf8]{inputenc}
\usepackage[spanish]{babel}
\usepackage{xcolor}
\usepackage{amsmath, amssymb}
\usepackage{tcolorbox}
\tcbset{
    colback=cyan!10, % Fondo celeste suave
    colframe=cyan!50!black, % Borde más oscuro
    boxrule=0.8pt,
    arc=3pt,
    left=6pt,
    right=6pt,
    top=6pt,
    bottom=6pt
}


\title{PC1 - Tarea 1 -  Latex 25-2}
\author{segundo }
\date{August 2025}

\begin{document}

\maketitle

\input{capitulos/cap5-5_1-Ninho}
\input{capitulos/7-7.1/cap7-7_1-Soto}
\end{document}
\documentclass[12pt]{article}
\usepackage[utf8]{inputenc}
\usepackage{tcolorbox}
\tcbset{colback=cyan!10!white, colframe=cyan!50!black, boxrule=0.8pt, arc=3mm, left=5pt, right=5pt, top=5pt, bottom=5pt}

\begin{document}

\section{Clases de Sistemas Híbridos}

\begin{tcolorbox}
Los sistemas híbridos combinan dinámicas continuas con eventos discretos. 
Existen varias clases importantes que capturan diferentes niveles de complejidad y dominios de aplicación. 
Aquí se destacan tres clases representativas.
\end{tcolorbox}

\subsection{Autómatas Temporizados}
\begin{tcolorbox}
Los autómatas temporizados extienden a las máquinas de estados finitos con uno o más \emph{relojes}.  
Estos relojes evolucionan linealmente con el tiempo y pueden reiniciarse al producirse una transición.  
Son útiles para modelar sistemas donde la corrección depende de restricciones temporales, 
como protocolos de comunicación, controladores con retardos temporales o planificadores de tareas.  
\end{tcolorbox}

Formalmente, un autómata temporizado consiste en estados, transiciones y un conjunto de relojes de valor real 
que deben cumplir guardas e invariantes.

\subsection{Dinámicas de Orden Superior}
\begin{tcolorbox}
Mientras que los autómatas temporizados sólo capturan el paso del tiempo, muchos sistemas físicos 
requieren el modelado de dinámicas continuas de orden superior.  
Por ejemplo, una masa conectada a un resorte presenta dinámicas de segundo orden.  
Cuando dichas masas colisionan, las ecuaciones que las gobiernan pueden cambiar abruptamente, 
produciendo \emph{dinámicas continuas por tramos}.  
\end{tcolorbox}

Esta clase de sistemas híbridos combina ecuaciones diferenciales con transiciones discretas, 
representando fenómenos como impactos mecánicos, circuitos conmutados o reacciones químicas.

\subsection{Control Supervisor}
\begin{tcolorbox}
En el control supervisor, un controlador discreto supervisa una planta continua.  
La parte discreta determina qué ley de control o modo debe estar activo, 
dependiendo de condiciones o eventos.  
Este enfoque es ampliamente utilizado en sistemas de ingeniería como 
la automatización industrial, los sistemas de seguridad automotriz y la robótica.  
\end{tcolorbox}

Ilustra cómo los modelos híbridos proporcionan una forma natural de integrar 
la toma de decisiones lógica con el control en retroalimentación continua.

\subsection*{Resumen}
\begin{tcolorbox}
Los sistemas híbridos abarcan modelos que combinan dinámicas continuas y discretas.  
Los autómatas temporizados ofrecen una base para razonar sobre el tiempo, 
las dinámicas de orden superior capturan procesos físicos más realistas, 
y el control supervisor proporciona un método estructurado para combinar la lógica con el control.  
\end{tcolorbox}

En conjunto, estas clases forman una base para modelar sistemas ciberfísicos complejos.

\end{document}
