\documentclass[12pt]{article}
\usepackage{graphicx} % Required for inserting images

\usepackage{cite}

\usepackage{amsmath}
\usepackage{circuitikz}
\usepackage{booktabs}
\usepackage{siunitx}
\usepackage{afterpage}
\usepackage{amsmath,amssymb,amsfonts}
\usepackage{algorithmic}
\usepackage{graphicx}
\usepackage{textcomp}
\usepackage{xcolor}
\usepackage{amssymb}
\usepackage{siunitx}
\usepackage{float}
\PassOptionsToPackage{hyphens}{url}\usepackage{hyperref}
\usepackage{cleveref}
\usepackage[utf8]{inputenc}
\usepackage[right]{lineno}
\usepackage{csquotes}
\usepackage{booktabs}
\usepackage{longtable}
\usepackage{adjustbox}
\usepackage{titlesec}
\usepackage{tikz,xcolor,pgfplots}
\usepackage{amsthm}
\theoremstyle{definition}
\newtheorem{example}{Ejemplo}


\usepackage{mdframed}
\usepackage[utf8]{inputenc}
\usepackage[spanish]{babel}
\usepackage{xcolor}
\usepackage{amsmath, amssymb}
\usepackage{tcolorbox}
\tcbset{
    colback=cyan!10, % Fondo celeste suave
    colframe=cyan!50!black, % Borde más oscuro
    boxrule=0.8pt,
    arc=3pt,
    left=6pt,
    right=6pt,
    top=6pt,
    bottom=6pt
}


\title{PC1 - Tarea 1 -  Latex 25-2}
\author{segundo }
\date{August 2025}

\begin{document}

\maketitle

\documentclass[12pt,a4paper]{article}

% Paquetes necesarios
\usepackage{setspace}          % Espaciado personalizado
\usepackage[spanish]{babel}    % Español
\usepackage{amsmath, amssymb}  % Símbolos matemáticos
\usepackage{tcolorbox}         % Recuadros de color
\usepackage{tikz}              % Diagramas de bloque
\usepackage{geometry}          % Márgenes
\usepackage{fancyhdr}          %Encabezados y pie de página
\usepackage{xcolor}

\usetikzlibrary{arrows.meta, positioning,shapes.geometric} 

\geometry{margin=3cm}
\setstretch{1.4}
\setlength{\parskip}{4pt} 
\pagestyle{empty}

\begin{document}

\thispagestyle{fancy}
\fancyhf{} 
\fancyhead[L]{\textit{\textcolor{black}{2.2 MODELO DE ACTORES}}}
\fancyfoot[R]{\textcolor{blue}{\textit{Lee \& Seshia, Introducción a los Sistemas Embebidos}}} 
\fancyfoot[L]{24}

\begin{tcolorbox}[colback=blue!10,colframe=blue!40!black]
El torque es la suma del torque causado por el rotor principal y el causado por el rotor de cola. 
Cuando estos se encuentran perfectamente balanceados, esa suma es cero. 
La salida de nuestro sistema será la velocidad angular $\dot{\theta}_y$ alrededor del eje $y$. 
La versión reducida dimensionalmente de (2.2) puede escribirse como
\[
\ddot{\theta}_y(t) = \frac{T_y(t)}{I_{yy}}.
\]
Integrando ambos lados, obtenemos la salida $\dot{\theta}$ como función de la entrada $T_y$,

\[
\dot{\theta}_y(t) = \dot{\theta}_y(0) + \frac{1}{I_{yy}} \int_{0}^{t} T_y(\tau) \, d\tau. \tag{2.4}
\]
\end{tcolorbox}

La observación crítica de este ejemplo es que, si eligiéramos modelar el helicóptero mediante, por ejemplo, 
dejando que $\mathbf{x}: \mathbb{R} \to \mathbb{R}^3$ represente la posición absoluta en el espacio de la cola del helicóptero, 
terminaríamos con un modelo mucho más complicado. 
El diseño del sistema de control también sería mucho más difícil.

\subsection*{2.2. Modelos de Actores}
En la sección anterior, un modelo de un sistema físico se da mediante una ecuación diferencial o integral 
que relaciona las señales de entrada (fuerza o torque) con señales de salida (posición, orientación, velocidad 
o velocidad de rotación). Tal sistema físico puede verse como un componente dentro de un sistema mayor. 
En particular, un \textbf{sistema en tiempo continuo} (uno que opera sobre \textit{señales de tiempo continuo}) 
puede modelarse mediante una caja con un \textbf{puerto de entrada} y un puerto de salida de la siguiente forma:

\begin{center}
\begin{tikzpicture}
% Defino el bloque central
\node[draw, minimum width=2.5cm, minimum height=1.5cm] (block)
    {\shortstack{\itshape parameters:\\ $p, q$}};
\node[above=2pt of block] {$S$};
% Flechas
\draw[-{Stealth[length=3mm]}] (-2,0) node[left] {$x$} -- (block.west);
\draw[-{Stealth[length=3mm]}] (block.east) -- (2,0) node[right] {$y$};
\end{tikzpicture}
\end{center}

donde la señal de entrada $x$ y la señal de salida $y$ son funciones de la forma

\[
x: \mathbb{R} \to \mathbb{R}, \quad y: \mathbb{R} \to \mathbb{R}.
\]
\newpage

\thispagestyle{fancy}
\fancyhf{} 
\fancyhead[R]{\textit{\textcolor{black}{2. DINÁMICA CONTINUA}}}
\fancyfoot[L]{\textcolor{blue}{\textit{Lee \& Seshia, Introducción a los Sistemas Embebidos}}} 
\fancyfoot[R]{25}

Aquí el dominio representa el \textit{tiempo} y el codominio representa el valor de la señal en un instante particular. 
El dominio $\mathbb{R}$ puede ser reemplazado por $\mathbb{R}_+$, los reales no negativos, si deseamos modelar explícitamente un sistema que comienza a existir y empieza a operar en un momento particular en el tiempo.  

El modelo del sistema es una función de la forma

\[
S: X \to Y, \tag{2.5}
\]
\setcounter{footnote}{2}
donde $X = Y = \mathbb{R}^{\mathbb{R}}$, el conjunto de funciones que mapean los reales en los reales, como $x$ e $y$ arriba.\footnote{Como se explica en el Apéndice A, la notación $\mathbb{R}^{\mathbb{R}}$ (que también puede escribirse $(\mathbb{R} \to \mathbb{R})$) representa el conjunto de todas las funciones con dominio $\mathbb{R}$ y codominio $\mathbb{R}$.} 

La función $S$ puede depender de parámetros del sistema, en cuyo caso los parámetros pueden mostrarse opcionalmente en la caja, y también opcionalmente incluirse en la notación de función.  
Por ejemplo, en la figura anterior, si hay parámetros $p$ y $q$, podríamos escribir la función del sistema como $S_{p,q}$ o incluso $S(p,q)$, teniendo en cuenta que ambas notaciones representan funciones de la forma en (2.5).  
Una caja como la de arriba, donde las entradas son funciones y las salidas son funciones, se llama un \textbf{actor}.  

\begin{tcolorbox}[colback=blue!10,colframe=blue!40!black]
Ejemplo 2.2: El modelo de actor para el helicóptero del ejemplo 2.1 puede representarse de la siguiente forma:

\begin{center}
\begin{tikzpicture}
% Defino el bloque central
\node[draw, minimum width=2.5cm, minimum height=1.5cm] (block)
    {\shortstack{$I_{yy}$ \\ $\dot{\theta}_y(0)$}};
\node[above=2pt of block] {Helicopter};
% Flechas
\draw[-{Stealth[length=3mm]}] (-2,0) node[left] {$T_y$} -- (block.west);
\draw[-{Stealth[length=3mm]}] (block.east) -- (2,0) node[right] {$\dot{\theta}_y$};
\end{tikzpicture}
\end{center}

La entrada y la salida son ambas funciones de tiempo continuo. 
Los parámetros del actor son la velocidad angular inicial $\dot{\theta}_y(0)$ y el momento de inercia $I_{yy}$.  
La función del actor está definida por (2.4).
\end{tcolorbox}

Los modelos de actores son composables. En particular, dados dos actores $S_1$ y $S_2$, 
podemos formar una \textbf{composición en cascada} de la siguiente manera:
\newpage

\thispagestyle{fancy}
\fancyhf{} 
\fancyhead[L]{\textit{\textcolor{black}{2.2 MODELO DE ACTORES}}}
\fancyfoot[R]{\textcolor{blue}{\textit{Lee \& Seshia, Introducción a los Sistemas Embebidos}}} 
\fancyfoot[L]{26}

\begin{center}
\begin{tikzpicture}
% Defino el bloque central
\node[draw, minimum width=2cm, minimum height=1.5cm] (S1) {};
\node[draw, minimum width=2cm, minimum height=1.5cm, right=of S1] (S2) {};
\node[above=2pt of S1] {$S_1$};
\node[above=2pt of S2] {$S_2$};
% Flechas
\draw[-{Stealth[length=3mm]}] (-2,0) node[left] {$x_1$} -- (S1.west);
\draw[-{Stealth[length=3mm]}] (S1.east) -- node[midway,above] {$y_1, x_2$} (S2.west);
\draw[-{Stealth[length=3mm]}] (S2.east) -- (5,0) node[right] {$y_2$};

\end{tikzpicture}
\end{center}

En el diagrama, la “conexión” entre la salida de $S_1$ y la entrada de $S_2$ significa precisamente que $y_1 = x_2$, o dicho más formalmente,
\[
\forall t \in \mathbb{R}, \quad y_1(t) = x_2(t).
\]
\begin{tcolorbox}[colback=blue!10,colframe=blue!40!black]
Ejemplo 2.3: El modelo de actor para el helicóptero puede representarse como una \textbf{composición en cascada} de dos actores de la siguiente forma:

\begin{center}
\begin{tikzpicture}
\node[draw, minimum width=6.2cm, minimum height=3cm, anchor=west] (frame) at (0,0) {};
\node[above=2pt of frame]{Helicopter};
\node[draw, isosceles triangle, isosceles triangle apex angle=60,
        minimum height=1.5cm, minimum width=1cm,
        shape border rotate=0, anchor=west, xshift=0.4cm] (scale) at (frame.west) {$a$};
\node[above=2pt of scale] {Scale};

\node[draw, minimum width=1.8cm, minimum height=1.5cm, right=2cm of scale] (int) {$\int i$};
\node[above=2pt of int] {Integrator};
% Flechas
\draw[-{Stealth[length=3mm]}] 
    (-1,0) node[left] {$x_1$} -- (scale.west);
\draw[-{Stealth[length=3mm]}] 
    (scale.east) -- node[midway,above] {$y_1, x_2$} (int.west);
\draw[-{Stealth[length=3mm]}] 
    (int.east) -- (7,0) node[right] {$y_2$};

\end{tikzpicture}
\end{center}

El actor de la izquierda representa un actor de \textbf{Escala}, parametrizado por la constante $a$, definida por

\[
\forall t \in \mathbb{R}, \quad y_1(t) = a x_1(t). \tag{2.6}
\]

De manera más compacta, podemos escribir $y_1 = a x_1$, donde se entiende que el producto de un escalar $a$ y una función $x_1$ se interpreta como en (2.6).  
El actor de la derecha representa un \textbf{Integrador} parametrizado por el valor inicial $i$, definido por

\[
\forall t \in \mathbb{R}, \quad y_2(t) = i + \int_0^t x_2(\tau) \, d\tau.
\]

Si damos a los parámetros los valores $a = \tfrac{1}{I_{yy}}$ e $i = \dot{\theta}_y(0)$, vemos que este sistema representa (2.4), donde la entrada $x_1 = T_y$ es el torque y la salida $y_2 = \dot{\theta}_y$ es la velocidad angular.
\end{tcolorbox}

\newpage


\thispagestyle{fancy}
\fancyhf{} 
\fancyhead[R]{\textit{\textcolor{black}{2. DINÁMICA CONTINUA}}}
\fancyfoot[L]{\textcolor{blue}{\textit{Lee \& Seshia, Introducción a los Sistemas Embebidos}}} 
\fancyfoot[R]{27}

En la figura anterior, hemos personalizado los \textit{iconos}, que son los cuadros que representan 
a los actores. Estos actores particulares (\textit{escalador} e \textit{integrador}) son bloques 
de construcción particularmente útiles para elaborar modelos de dinámicas físicas, de modo que 
asignarles notaciones visuales reconocibles es útil.
Podemos tener actores que tengan múltiples señales de entrada y/o múltiples señales de salida. 
Estos se representan de manera similar, como en el siguiente ejemplo, que tiene dos señales de 
entrada y una señal de salida:

\begin{center}
\begin{tikzpicture}[baseline=(current bounding box.center)]
\node[draw, minimum width=2cm, minimum height=1.2cm] (block) {$S$};
\node[left=of block.west, xshift=-1cm, yshift=0.4cm] (x1) {$x_1$};
\node[left=of block.west, xshift=-1cm, yshift=-0.4cm] (x2) {$x_2$};
\node[right=of block.east, xshift=1cm] (y) {$y$};
\draw[-{Stealth[length=3mm]}] (x1.east) -- (block.west |- x1.east);
\draw[-{Stealth[length=3mm]}] (x2.east) -- (block.west |- x2.east);
\draw[-{Stealth[length=3mm]}] (block.east) -- (y.west);
\end{tikzpicture}
\end{center}
Un bloque de construcción particularmente útil con esta forma es un \textit{sumador de señales}, 
definido por
\[
  \forall t \in \mathbb{R}, \quad y(t) = x_1(t) + x_2(t).
\]
Esto a menudo se representará mediante un ícono personalizado como el siguiente:
\begin{center}
\begin{tikzpicture}[baseline=(current bounding box.center)]
\node[draw, circle, minimum size=1cm] (sum) {$+$};
\node[left=1cm of sum] (x1) {$x_1$};
\node[below=1cm of sum] (x2) {$x_2$};
\node[right=1cm of sum] (y) {$y$};
\draw[-{Stealth[length=3mm]}] (x1.east) -- (sum.west);
\draw[-{Stealth[length=3mm]}] (x2.north) -- (sum.south);
\draw[-{Stealth[length=3mm]}] (sum.east) -- (y.west);
\end{tikzpicture}
\end{center}
A veces, una de las entradas se restará en lugar de sumarse; en ese caso, el ícono se 
personaliza aún más con un signo menos cerca de esa entrada, como se muestra a continuación:
\begin{center}
\begin{tikzpicture}[baseline=(current bounding box.center)]
\node[draw, circle, minimum size=1cm] (sum) {$+$};
\node[left=1cm of sum] (x1) {$x_1$};
\node[below=1cm of sum] (x2) {$x_2$};
\node[right=1cm of sum] (y) {$y$};
\draw[-{Stealth[length=3mm]}] (x1.east) -- (sum.west);
\draw[-{Stealth[length=3mm]}] (x2.north) -- (sum.south);
\draw[-{Stealth[length=3mm]}] (sum.east) -- (y.west);
\node[below right =2pt of sum] {$-$};
\end{tikzpicture}
\end{center}
Este actor representa una función \(S : (\mathbb{R} \to \mathbb{R})^2 \to (\mathbb{R} \to \mathbb{R})\) dada por
\[
\forall t \in \mathbb{R}, \; \forall x_1, x_2 \in (\mathbb{R} \to \mathbb{R}), \quad
\big( S(x_1, x_2) \big)(t) = y(t) = x_1(t) - x_2(t).
\]
Nótese la notación cuidadosa. \(S(x_1, x_2)\) es una función en \(\mathbb{R}^\mathbb{R}\). 
Por lo tanto, puede evaluarse en un \(t \in \mathbb{R}\).



\end{document}


\documentclass[12pt,a4paper]{article}

% Paquetes de idioma y codificación
\usepackage[utf8]{inputenc}
\usepackage[T1]{fontenc}
\usepackage[english]{babel}

% Márgenes y formato de página
\usepackage{geometry}
\geometry{a4paper, margin=2.5cm}

% Paquetes matemáticos
\usepackage{amsmath, amssymb, amsthm}

% Mejor manejo de fuentes
\usepackage{lmodern}

% Definición de entorno "example"
\newtheorem{example}{Ejemplo}[section]

% Títulos
\usepackage{titlesec}
\titleformat{\section}{\large\bfseries}{\thesection}{1em}{}
\titleformat{\subsection}{\normalsize\bfseries}{\thesubsection}{1em}{}

% Documento
\begin{document}
\subsection{Propiedades de los Sistemas}

En esta sección, consideramos una serie de propiedades que los actores y los sistemas que ellos componen pueden tener, incluyendo causalidad, ausencia de memoria, linealidad, invariancia en el tiempo y estabilidad.

\subsubsection{Sistemas Causales}

Intuitivamente, un sistema es causal si su salida depende solo de las entradas presentes y pasadas. Sin embargo, precisar esta noción es un poco complicado. Lo hacemos dando primero una notación para “entradas presentes y pasadas.” Consideremos una señal de tiempo continuo $x : \mathbb{R}\rightarrow A$, para algún conjunto $A$.
Sea $x|_{t\leq\tau}$ una función llamada la restricción en el tiempo que solo está definida para tiempos $t \leq \tau$, y donde está definida, $x|_{t\leq\tau}(t) = x(t)$. Así, si $x$ es una entrada a un sistema, entonces $x|_{t\leq\tau}$ son las “entradas presentes y pasadas” en el tiempo $\tau$.

Consideremos un sistema de tiempo continuo $S : X \rightarrow Y$, donde $X = A^{\mathbb{R}}$ y $Y = B^{\mathbb{R}}$ para algunos conjuntos $A$ y $B$. Este sistema es causal si para todo $x_1, x_2 \in X$ y $\tau \in \mathbb{R}$,

\[
x_1|_{t\leq\tau} = x_2|_{t\leq\tau} \Rightarrow S(x_1)|_{t\leq\tau} = S(x_2)|_{t\leq\tau}
\]

Es decir, el sistema es causal si para dos posibles entradas $x_1$ y $x_2$ que son idénticas hasta (e incluyendo) el tiempo $\tau$, las salidas son idénticas hasta (e incluyendo) el tiempo $\tau$. Todos los sistemas que hemos considerado hasta ahora son causales.

Un sistema es estrictamente causal si para todo $x_1, x_2 \in X$ y $\tau \in \mathbb{R}$,

\[
x_1|_{t<\tau} = x_2|_{t<\tau} \Rightarrow S(x_1)|_{t\leq\tau} = S(x_2)|_{t\leq\tau}
\]

Es decir, el sistema es estrictamente causal si para dos posibles entradas $x_1$ y $x_2$ que son idénticas hasta (y sin incluir) el tiempo $\tau$, las salidas son idénticas hasta (e incluyendo) el tiempo $\tau$.
La salida en el tiempo $t$ de un sistema estrictamente causal no depende de su entrada en el tiempo $t$.

Depende solo de las entradas pasadas. Un sistema estrictamente causal, por supuesto, también es causal. El Integrador es estrictamente causal. El sumador no es estrictamente causal, pero sí es causal. Los actores estrictamente causales son útiles para construir sistemas de retroalimentación.

\subsubsection{Sistemas sin Memoria}

Intuitivamente, un sistema tiene memoria si la salida depende no solo de las entradas presentes, sino también de las entradas pasadas (o entradas futuras, si el sistema no es causal). Consideremos un sistema de tiempo continuo $S : X \rightarrow Y$, donde $X = A^{\mathbb{R}}$ y $Y = B^{\mathbb{R}}$ para algunos conjuntos $A$ y $B$. Formalmente,
este sistema es sin memoria si existe una función $f : A \rightarrow B$ tal que para todo $x \in X$,

\[
(S(x))(t) = f(x(t))
\]

para todo $t \in \mathbb{R}$. Es decir, la salida $(S(x))(t)$ en el tiempo $t$ depende solo de la entrada $x(t)$ en el tiempo $t$.

El Integrador considerado arriba no es sin memoria, pero el sumador sí lo es. El Ejercicio 2 muestra que si un sistema es estrictamente causal y sin memoria, entonces su salida es constante para todas las entradas.

\subsubsection{Linealidad e Invariancia en el Tiempo}

Los sistemas que son lineales e invariantes en el tiempo (LTI) tienen propiedades matemáticas particularmente útiles. Gran parte de la teoría de sistemas de control depende de estas propiedades. Estas propiedades forman el cuerpo principal de los cursos de señales y sistemas, y están más allá del alcance de este texto. Pero ocasionalmente aprovecharemos versiones simples de las propiedades, por lo que es útil determinar cuándo un sistema es LTI.

Un sistema $S : X \rightarrow Y$, donde $X$ y $Y$ son conjuntos de señales, es lineal si satisface la propiedad de superposición:

\[
\forall x_1, x_2 \in X \text{ y } \forall a, b \in \mathbb{R}, \quad S(ax_1 + bx_2) = aS(x_1) + bS(x_2).
\]

Es fácil ver que el sistema de helicóptero definido en el Ejemplo 2.1 es lineal si y solo si la velocidad angular inicial $\dot{\theta}_y(0) = 0$ (ver Ejercicio 3).

Más generalmente, es fácil ver que un integrador como se define en el Ejemplo 2.3 es lineal si y solo si el valor inicial $i = 0$, que el actor Escalador es siempre lineal, y que la cascada de dos actores lineales es lineal. Podemos extender trivialmente la definición de linealidad a actores con más de una señal de entrada o salida y luego determinar que el sumador también es lineal.

Para definir invariancia en el tiempo, primero definimos un actor especializado de tiempo continuo llamado
retardo. Sea $D_\tau : X \rightarrow Y$, donde $X$ y $Y$ son conjuntos de señales de tiempo continuo, definido por

\[
\forall x \in X \text{ y } \forall t \in \mathbb{R}, \quad (D_\tau(x))(t) = x(t-\tau). \tag{2.7}
\]

Aquí, $\tau$ es un parámetro del actor de retardo. Un sistema $S : X \rightarrow Y$ es invariante en el tiempo si

\[
\forall x \in X \text{ y } \forall \tau \in \mathbb{R}, \quad S(D_\tau(x)) = D_\tau(S(x)).
\]

El sistema de helicóptero definido en el Ejemplo 2.1 y (2.4) no es invariante en el tiempo. Una variante menor, sin embargo, sí lo es:

\[
\dot{\theta}_y(t) = \frac{1}{I_{yy}} \int_{-\infty}^{t} T_y(\tau) d\tau
\]

Esta versión no permite una rotación angular inicial.

Un sistema lineal invariante en el tiempo (LTI) es un sistema que es tanto lineal como invariante en el tiempo.
Un objetivo importante en el modelado de dinámicas físicas es elegir un modelo LTI siempre que sea posible. Si una aproximación razonable resulta en un modelo LTI, vale la pena hacer esta aproximación. No siempre es fácil determinar si la aproximación es razonable, o encontrar modelos para los que la aproximación sea razonable. A menudo es fácil construir modelos más complicados de lo necesario (ver Ejercicio 4).

\subsubsection{Estabilidad}

Se dice que un sistema es estable de entrada acotada-salida acotada (BIBO estable o simplemente estable) si la señal de salida está acotada para todas las señales de entrada que están acotadas.

Consideremos un sistema de tiempo continuo con entrada $w$ y salida $v$. La entrada está acotada si existe un número real $A < \infty$ tal que $|w(t)| \leq A$ para todo $t \in \mathbb{R}$. La salida está acotada si existe un número real $B < \infty$ tal que $|v(t)| \leq B$ para todo $t \in \mathbb{R}$. El sistema es estable si para cualquier entrada acotada por algún $A$, existe alguna cota $B$ en la salida.

\begin{example}[]
Ahora es fácil ver que el sistema de helicóptero desarrollado en el Ejemplo 2.1 es inestable. Sea la entrada $T_y = u$, donde $u$ es el escalón unitario, dado por

\[
\forall t \in \mathbb{R}, u(t) =
\begin{cases}
0, & t < 0 \\
1, & t \geq 0
\end{cases} \tag{2.8}
\]

Esto significa que antes del tiempo cero, no se aplica torque al sistema, y comenzando en el tiempo cero, aplicamos un torque de magnitud unitaria. Esta entrada está claramente acotada. Nunca excede uno en magnitud. Sin embargo, la salida crece sin límite. En la práctica, un helicóptero usa un sistema de retroalimentación para determinar cuánto torque aplicar en el rotor de cola para mantener recto el cuerpo del helicóptero. Estudiaremos cómo hacer eso a continuación.
\end{example}

\end{document}

\end{document}
\documentclass[12pt]{article}
\usepackage[utf8]{inputenc}
\usepackage{tcolorbox}
\tcbset{colback=cyan!10!white, colframe=cyan!50!black, boxrule=0.8pt, arc=3mm, left=5pt, right=5pt, top=5pt, bottom=5pt}

\begin{document}

\section{Clases de Sistemas Híbridos}

\begin{tcolorbox}
Los sistemas híbridos combinan dinámicas continuas con eventos discretos. 
Existen varias clases importantes que capturan diferentes niveles de complejidad y dominios de aplicación. 
Aquí se destacan tres clases representativas.
\end{tcolorbox}

\subsection{Autómatas Temporizados}
\begin{tcolorbox}
Los autómatas temporizados extienden a las máquinas de estados finitos con uno o más \emph{relojes}.  
Estos relojes evolucionan linealmente con el tiempo y pueden reiniciarse al producirse una transición.  
Son útiles para modelar sistemas donde la corrección depende de restricciones temporales, 
como protocolos de comunicación, controladores con retardos temporales o planificadores de tareas.  
\end{tcolorbox}

Formalmente, un autómata temporizado consiste en estados, transiciones y un conjunto de relojes de valor real 
que deben cumplir guardas e invariantes.

\subsection{Dinámicas de Orden Superior}
\begin{tcolorbox}
Mientras que los autómatas temporizados sólo capturan el paso del tiempo, muchos sistemas físicos 
requieren el modelado de dinámicas continuas de orden superior.  
Por ejemplo, una masa conectada a un resorte presenta dinámicas de segundo orden.  
Cuando dichas masas colisionan, las ecuaciones que las gobiernan pueden cambiar abruptamente, 
produciendo \emph{dinámicas continuas por tramos}.  
\end{tcolorbox}

Esta clase de sistemas híbridos combina ecuaciones diferenciales con transiciones discretas, 
representando fenómenos como impactos mecánicos, circuitos conmutados o reacciones químicas.

\subsection{Control Supervisor}
\begin{tcolorbox}
En el control supervisor, un controlador discreto supervisa una planta continua.  
La parte discreta determina qué ley de control o modo debe estar activo, 
dependiendo de condiciones o eventos.  
Este enfoque es ampliamente utilizado en sistemas de ingeniería como 
la automatización industrial, los sistemas de seguridad automotriz y la robótica.  
\end{tcolorbox}

Ilustra cómo los modelos híbridos proporcionan una forma natural de integrar 
la toma de decisiones lógica con el control en retroalimentación continua.

\subsection*{Resumen}
\begin{tcolorbox}
Los sistemas híbridos abarcan modelos que combinan dinámicas continuas y discretas.  
Los autómatas temporizados ofrecen una base para razonar sobre el tiempo, 
las dinámicas de orden superior capturan procesos físicos más realistas, 
y el control supervisor proporciona un método estructurado para combinar la lógica con el control.  
\end{tcolorbox}

En conjunto, estas clases forman una base para modelar sistemas ciberfísicos complejos.

\end{document}
