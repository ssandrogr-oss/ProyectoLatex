\documentclass[12pt,a4paper]{article}

% Paquetes necesarios
\usepackage{setspace}          % Espaciado personalizado
\usepackage[spanish]{babel}    % Español
\usepackage{amsmath, amssymb}  % Símbolos matemáticos
\usepackage{tcolorbox}         % Recuadros de color
\usepackage{tikz}              % Diagramas de bloque
\usepackage{geometry}          % Márgenes
\usepackage{fancyhdr}          %Encabezados y pie de página
\usepackage{xcolor}

\usetikzlibrary{arrows.meta, positioning,shapes.geometric} 

\geometry{margin=3cm}
\setstretch{1.4}
\setlength{\parskip}{4pt} 
\pagestyle{empty}

\begin{document}

\thispagestyle{fancy}
\fancyhf{} 
\fancyhead[L]{\textit{\textcolor{black}{2.2 MODELO DE ACTORES}}}
\fancyfoot[R]{\textcolor{blue}{\textit{Lee \& Seshia, Introducción a los Sistemas Embebidos}}} 
\fancyfoot[L]{24}

\begin{tcolorbox}[colback=blue!10,colframe=blue!40!black]
El torque es la suma del torque causado por el rotor principal y el causado por el rotor de cola. 
Cuando estos se encuentran perfectamente balanceados, esa suma es cero. 
La salida de nuestro sistema será la velocidad angular $\dot{\theta}_y$ alrededor del eje $y$. 
La versión reducida dimensionalmente de (2.2) puede escribirse como
\[
\ddot{\theta}_y(t) = \frac{T_y(t)}{I_{yy}}.
\]
Integrando ambos lados, obtenemos la salida $\dot{\theta}$ como función de la entrada $T_y$,

\[
\dot{\theta}_y(t) = \dot{\theta}_y(0) + \frac{1}{I_{yy}} \int_{0}^{t} T_y(\tau) \, d\tau. \tag{2.4}
\]
\end{tcolorbox}

La observación crítica de este ejemplo es que, si eligiéramos modelar el helicóptero mediante, por ejemplo, 
dejando que $\mathbf{x}: \mathbb{R} \to \mathbb{R}^3$ represente la posición absoluta en el espacio de la cola del helicóptero, 
terminaríamos con un modelo mucho más complicado. 
El diseño del sistema de control también sería mucho más difícil.

\subsection*{2.2. Modelos de Actores}
En la sección anterior, un modelo de un sistema físico se da mediante una ecuación diferencial o integral 
que relaciona las señales de entrada (fuerza o torque) con señales de salida (posición, orientación, velocidad 
o velocidad de rotación). Tal sistema físico puede verse como un componente dentro de un sistema mayor. 
En particular, un \textbf{sistema en tiempo continuo} (uno que opera sobre \textit{señales de tiempo continuo}) 
puede modelarse mediante una caja con un \textbf{puerto de entrada} y un puerto de salida de la siguiente forma:

\begin{center}
\begin{tikzpicture}
% Defino el bloque central
\node[draw, minimum width=2.5cm, minimum height=1.5cm] (block)
    {\shortstack{\itshape parameters:\\ $p, q$}};
\node[above=2pt of block] {$S$};
% Flechas
\draw[-{Stealth[length=3mm]}] (-2,0) node[left] {$x$} -- (block.west);
\draw[-{Stealth[length=3mm]}] (block.east) -- (2,0) node[right] {$y$};
\end{tikzpicture}
\end{center}

donde la señal de entrada $x$ y la señal de salida $y$ son funciones de la forma

\[
x: \mathbb{R} \to \mathbb{R}, \quad y: \mathbb{R} \to \mathbb{R}.
\]
\newpage

\thispagestyle{fancy}
\fancyhf{} 
\fancyhead[R]{\textit{\textcolor{black}{2. DINÁMICA CONTINUA}}}
\fancyfoot[L]{\textcolor{blue}{\textit{Lee \& Seshia, Introducción a los Sistemas Embebidos}}} 
\fancyfoot[R]{25}

Aquí el dominio representa el \textit{tiempo} y el codominio representa el valor de la señal en un instante particular. 
El dominio $\mathbb{R}$ puede ser reemplazado por $\mathbb{R}_+$, los reales no negativos, si deseamos modelar explícitamente un sistema que comienza a existir y empieza a operar en un momento particular en el tiempo.  

El modelo del sistema es una función de la forma

\[
S: X \to Y, \tag{2.5}
\]
\setcounter{footnote}{2}
donde $X = Y = \mathbb{R}^{\mathbb{R}}$, el conjunto de funciones que mapean los reales en los reales, como $x$ e $y$ arriba.\footnote{Como se explica en el Apéndice A, la notación $\mathbb{R}^{\mathbb{R}}$ (que también puede escribirse $(\mathbb{R} \to \mathbb{R})$) representa el conjunto de todas las funciones con dominio $\mathbb{R}$ y codominio $\mathbb{R}$.} 

La función $S$ puede depender de parámetros del sistema, en cuyo caso los parámetros pueden mostrarse opcionalmente en la caja, y también opcionalmente incluirse en la notación de función.  
Por ejemplo, en la figura anterior, si hay parámetros $p$ y $q$, podríamos escribir la función del sistema como $S_{p,q}$ o incluso $S(p,q)$, teniendo en cuenta que ambas notaciones representan funciones de la forma en (2.5).  
Una caja como la de arriba, donde las entradas son funciones y las salidas son funciones, se llama un \textbf{actor}.  

\begin{tcolorbox}[colback=blue!10,colframe=blue!40!black]
Ejemplo 2.2: El modelo de actor para el helicóptero del ejemplo 2.1 puede representarse de la siguiente forma:

\begin{center}
\begin{tikzpicture}
% Defino el bloque central
\node[draw, minimum width=2.5cm, minimum height=1.5cm] (block)
    {\shortstack{$I_{yy}$ \\ $\dot{\theta}_y(0)$}};
\node[above=2pt of block] {Helicopter};
% Flechas
\draw[-{Stealth[length=3mm]}] (-2,0) node[left] {$T_y$} -- (block.west);
\draw[-{Stealth[length=3mm]}] (block.east) -- (2,0) node[right] {$\dot{\theta}_y$};
\end{tikzpicture}
\end{center}

La entrada y la salida son ambas funciones de tiempo continuo. 
Los parámetros del actor son la velocidad angular inicial $\dot{\theta}_y(0)$ y el momento de inercia $I_{yy}$.  
La función del actor está definida por (2.4).
\end{tcolorbox}

Los modelos de actores son composables. En particular, dados dos actores $S_1$ y $S_2$, 
podemos formar una \textbf{composición en cascada} de la siguiente manera:
\newpage

\thispagestyle{fancy}
\fancyhf{} 
\fancyhead[L]{\textit{\textcolor{black}{2.2 MODELO DE ACTORES}}}
\fancyfoot[R]{\textcolor{blue}{\textit{Lee \& Seshia, Introducción a los Sistemas Embebidos}}} 
\fancyfoot[L]{26}

\begin{center}
\begin{tikzpicture}
% Defino el bloque central
\node[draw, minimum width=2cm, minimum height=1.5cm] (S1) {};
\node[draw, minimum width=2cm, minimum height=1.5cm, right=of S1] (S2) {};
\node[above=2pt of S1] {$S_1$};
\node[above=2pt of S2] {$S_2$};
% Flechas
\draw[-{Stealth[length=3mm]}] (-2,0) node[left] {$x_1$} -- (S1.west);
\draw[-{Stealth[length=3mm]}] (S1.east) -- node[midway,above] {$y_1, x_2$} (S2.west);
\draw[-{Stealth[length=3mm]}] (S2.east) -- (5,0) node[right] {$y_2$};

\end{tikzpicture}
\end{center}

En el diagrama, la “conexión” entre la salida de $S_1$ y la entrada de $S_2$ significa precisamente que $y_1 = x_2$, o dicho más formalmente,
\[
\forall t \in \mathbb{R}, \quad y_1(t) = x_2(t).
\]
\begin{tcolorbox}[colback=blue!10,colframe=blue!40!black]
Ejemplo 2.3: El modelo de actor para el helicóptero puede representarse como una \textbf{composición en cascada} de dos actores de la siguiente forma:

\begin{center}
\begin{tikzpicture}
\node[draw, minimum width=6.2cm, minimum height=3cm, anchor=west] (frame) at (0,0) {};
\node[above=2pt of frame]{Helicopter};
\node[draw, isosceles triangle, isosceles triangle apex angle=60,
        minimum height=1.5cm, minimum width=1cm,
        shape border rotate=0, anchor=west, xshift=0.4cm] (scale) at (frame.west) {$a$};
\node[above=2pt of scale] {Scale};

\node[draw, minimum width=1.8cm, minimum height=1.5cm, right=2cm of scale] (int) {$\int i$};
\node[above=2pt of int] {Integrator};
% Flechas
\draw[-{Stealth[length=3mm]}] 
    (-1,0) node[left] {$x_1$} -- (scale.west);
\draw[-{Stealth[length=3mm]}] 
    (scale.east) -- node[midway,above] {$y_1, x_2$} (int.west);
\draw[-{Stealth[length=3mm]}] 
    (int.east) -- (7,0) node[right] {$y_2$};

\end{tikzpicture}
\end{center}

El actor de la izquierda representa un actor de \textbf{Escala}, parametrizado por la constante $a$, definida por

\[
\forall t \in \mathbb{R}, \quad y_1(t) = a x_1(t). \tag{2.6}
\]

De manera más compacta, podemos escribir $y_1 = a x_1$, donde se entiende que el producto de un escalar $a$ y una función $x_1$ se interpreta como en (2.6).  
El actor de la derecha representa un \textbf{Integrador} parametrizado por el valor inicial $i$, definido por

\[
\forall t \in \mathbb{R}, \quad y_2(t) = i + \int_0^t x_2(\tau) \, d\tau.
\]

Si damos a los parámetros los valores $a = \tfrac{1}{I_{yy}}$ e $i = \dot{\theta}_y(0)$, vemos que este sistema representa (2.4), donde la entrada $x_1 = T_y$ es el torque y la salida $y_2 = \dot{\theta}_y$ es la velocidad angular.
\end{tcolorbox}

\newpage


\thispagestyle{fancy}
\fancyhf{} 
\fancyhead[R]{\textit{\textcolor{black}{2. DINÁMICA CONTINUA}}}
\fancyfoot[L]{\textcolor{blue}{\textit{Lee \& Seshia, Introducción a los Sistemas Embebidos}}} 
\fancyfoot[R]{27}

En la figura anterior, hemos personalizado los \textit{iconos}, que son los cuadros que representan 
a los actores. Estos actores particulares (\textit{escalador} e \textit{integrador}) son bloques 
de construcción particularmente útiles para elaborar modelos de dinámicas físicas, de modo que 
asignarles notaciones visuales reconocibles es útil.
Podemos tener actores que tengan múltiples señales de entrada y/o múltiples señales de salida. 
Estos se representan de manera similar, como en el siguiente ejemplo, que tiene dos señales de 
entrada y una señal de salida:

\begin{center}
\begin{tikzpicture}[baseline=(current bounding box.center)]
\node[draw, minimum width=2cm, minimum height=1.2cm] (block) {$S$};
\node[left=of block.west, xshift=-1cm, yshift=0.4cm] (x1) {$x_1$};
\node[left=of block.west, xshift=-1cm, yshift=-0.4cm] (x2) {$x_2$};
\node[right=of block.east, xshift=1cm] (y) {$y$};
\draw[-{Stealth[length=3mm]}] (x1.east) -- (block.west |- x1.east);
\draw[-{Stealth[length=3mm]}] (x2.east) -- (block.west |- x2.east);
\draw[-{Stealth[length=3mm]}] (block.east) -- (y.west);
\end{tikzpicture}
\end{center}
Un bloque de construcción particularmente útil con esta forma es un \textit{sumador de señales}, 
definido por
\[
  \forall t \in \mathbb{R}, \quad y(t) = x_1(t) + x_2(t).
\]
Esto a menudo se representará mediante un ícono personalizado como el siguiente:
\begin{center}
\begin{tikzpicture}[baseline=(current bounding box.center)]
\node[draw, circle, minimum size=1cm] (sum) {$+$};
\node[left=1cm of sum] (x1) {$x_1$};
\node[below=1cm of sum] (x2) {$x_2$};
\node[right=1cm of sum] (y) {$y$};
\draw[-{Stealth[length=3mm]}] (x1.east) -- (sum.west);
\draw[-{Stealth[length=3mm]}] (x2.north) -- (sum.south);
\draw[-{Stealth[length=3mm]}] (sum.east) -- (y.west);
\end{tikzpicture}
\end{center}
A veces, una de las entradas se restará en lugar de sumarse; en ese caso, el ícono se 
personaliza aún más con un signo menos cerca de esa entrada, como se muestra a continuación:
\begin{center}
\begin{tikzpicture}[baseline=(current bounding box.center)]
\node[draw, circle, minimum size=1cm] (sum) {$+$};
\node[left=1cm of sum] (x1) {$x_1$};
\node[below=1cm of sum] (x2) {$x_2$};
\node[right=1cm of sum] (y) {$y$};
\draw[-{Stealth[length=3mm]}] (x1.east) -- (sum.west);
\draw[-{Stealth[length=3mm]}] (x2.north) -- (sum.south);
\draw[-{Stealth[length=3mm]}] (sum.east) -- (y.west);
\node[below right =2pt of sum] {$-$};
\end{tikzpicture}
\end{center}
Este actor representa una función \(S : (\mathbb{R} \to \mathbb{R})^2 \to (\mathbb{R} \to \mathbb{R})\) dada por
\[
\forall t \in \mathbb{R}, \; \forall x_1, x_2 \in (\mathbb{R} \to \mathbb{R}), \quad
\big( S(x_1, x_2) \big)(t) = y(t) = x_1(t) - x_2(t).
\]
Nótese la notación cuidadosa. \(S(x_1, x_2)\) es una función en \(\mathbb{R}^\mathbb{R}\). 
Por lo tanto, puede evaluarse en un \(t \in \mathbb{R}\).



\end{document}

